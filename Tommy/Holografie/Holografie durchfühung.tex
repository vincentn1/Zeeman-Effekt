\documentclass[10pt,a4paper]{article}
\usepackage[utf8]{inputenc}
\usepackage[english]{babel}
\usepackage[T1]{fontenc}
\usepackage{amsmath}
\usepackage{amsfonts}
\usepackage{amssymb}
\usepackage{subcaption}
\usepackage{makeidx}
\usepackage{graphicx}
\usepackage{fourier}
\usepackage{listings}
\usepackage{color}
\usepackage{hyperref}
\usepackage[left=2cm,right=2cm,top=2cm,bottom=2cm]{geometry}
\author{Tommy Müller, Marcus Dittrich, Vincent Noculak}
\title{Holografie}


\begin{document}
\section{Fragestellung}

Ziel des Versuchs ist der Aufbau einer Apparatur zur Erzeugung von Weißlichthologrammen. Es soll die Stabilität des Aufbaus überprüft werden und mehrere Hologramme erstellt werden. Sowie die Kohärenzlänge des Lasers ermittelt werde.

\section{Durchführung}


\subsection{Erstellung der Hologramme}

Als erstes haben wir den Aufbau zum erstellen der Hologramme verwendet. Das zu holografierende Objekt wurde auf einem, um 30° gekippten, Sockel in den Strahlengang gestellt. Unsere Objekte waren eine Schachfigur, Monopoly Hotel, Origami Figur und ein Happyhippo R2D2. Vor dem Sockel befand sich die Scheibe, die als Halterung für den holografischen Film diente. 
Der Laserstrahl ging zunächst durch die Scheibe und wurde dann vom Objekt gestreut. Die Objekte müssen sich für die Aufnahme möglichst nah am Film befinden, damit erstens sich das gesamte Objekt innerhalb der Kohärenzlänge des Lasers (wegen der Reflektion darf der Abstand nur die Hälfte der Kohärenzlänge sein, also ca. 5 cm) und zweitens damit möglichst viel Streulicht auf dem Film eingefangen werden kann. Die Verkippung der Sockels hat ebenfalls rein praktische Gründe: Zum einen wird dadurch Mehrfachreflektion eingeschränkt, die die Schärfe des Bildes beeinträchtigen könnte. Zweitens ist zu Beachten, dass beim späteren Betrachten des Licht im selben Winkel einfallen muss wie der Referenzstrahl bei der Aufnahme. Das Bild erscheint im gespiegelten Winkel. Würde der Referenzstrahl senkrecht auf den Film fallen, würde der Beobachter also später genau die Lichtquelle mit seinem Schatten verdecken. Schließlich lässt sich, falls die Verkippung genau im Brewsterwinkel ist, die Reflektion minimieren, sodass mehr Intensität für die Bilder zur Verfügung steht. Im Folgenden haben wir dann in vollständiger Dunkelheit den Film präpariert, ihn die Glasscheibe eingesetzt, und diese in die Halterung am Sockel eingeführt. Nach einer Zeit von mehreren Minuten zum Ausschwingen des Tisches haben wir dann je ein Bild mit 1s und ein Bild mit 2s Belichtungszeit gemacht, jeweils für alle Objekte.

\subsection{Entwicklung des Films}



Die Bilder wurden dann von uns in die Dunkelkammer zum Entwickeln gebracht und dort, wieder in absoluter Dunkelheit zunächst für zwei Minuten in einer Vitamin-C-Lösung (Ascorbinsäure und NaOH) entwickelt, kurz in Wasser gelegt, und schließlich in Chromschwefelsäure gebleicht. Danach wurden sie zum Trocken 1 Stunde in einen Ofen gehängt. Durch das Bleichen wird das Hologramm zu einem Phasenhologramm. 
Unmittelbar nach dem Fixieren ist der Film an den belichteten Stellen geschwärzt, was ein Amplitudenhologramm darstellt: die Schwärzung modifiziert die Intensität des Lichts beim Beobachten. Die Bleichlösung verändert den Film nun 
chemisch so, dass anstelle der Schwärzung der Brechungsindex und die Dicke des Filmmaterials verändert wird. Das Licht beim Beobachten wird jetzt nur noch in seiner Phase, nicht aber in seine Intensität verändert. F¨ur die von uns mit nur relativ kurzer Belichtungszeit hergestellten Hologramme ist nur ein solches Phasenhologramm sinnvoll, das Bild wäre ansonsten zu dunkel um sichtbar zu sein. Nach dem Trocknen waren die abgebildeten Objekte alle sehr deutlich dreidimensional sichtbar. Beim Trockenen ist die Gelatineschicht auf dem Film geschrumpft, was zu einer Verkleinerung des Gitters führt, sodass das Bild nun gr¨un erscheint, und nicht mehr im Rot des Lasers. Die Bilder mit längerer Belichtungszeit sind intensiver, allerdings auch unschärfer, wie man es erwartet.

\subsection{Kohärenzlänge}

Für die Bestimmung der Kohärenzlänge wurde zunächst ein Michelson-Interferometer aufgebaut. Dafür war viel Feingefühl nötig, um die Spiegel richtig zu positionieren und letztendlich ein Interferenzmuster zu erhalten. Das Interferenzmuster wurde auf einen Sensor abgebildet, das mit dem zur Verfügung stehenden Computer verbunden war. Das Interferometer wurde mit verschiedenen Spiegel-Abständen aufgebaut und jeweils mit Wackeln des Tisches ein wackelndes Interferenzmuster auf dem Sensor erzeugt. Mithilfe eines Programmes konnten über die Zeit die durchgehenden Amplitudenmaxima und -minima erfasst werden. Das Programm errechnete daraus die zugehörige Kontrastfunktion K. 


\subsection{Stabilität des Aufbaus}

Sodann haben wir das Interferenzmuster mit einem elektronischen Sensor erfasst und an den PC weitergeleitet, um Festzustellen, in welchem Maße sich Störungen auf den Aufbau auswirken würden. In einer Kurzzeitmessung haben wir Störungen wie lautes Sprechen, Schlagen der Tür, Aufstampfen und Berührungen des Tisches beobachtet. In den Ergebnissen der Kurzzeitmessung zeigt sich deutlich, dass Aufstampfen und die Benutzung der Tür einen wesentlichen Störeinfluss darstellen. Noch sehr viel empfindlicher reagiert der Aufbau auf Berühren des Tisches oder des Kastens, was zu wilden Ausschlägen auf den PC führt, die erst nach mehr längerer Zeit wieder abklingen. Dem gegenüber hat Reden in mäßiger Lautstärke nur einen geringeren Einfluss auf die Stabilität.

\section{Diskussion}

Wir konnten die Kohärenzlänge des Lasers graphisch ermitteln:   cm. 
Dies erscheint in der Größenordnung durchaus plausibel. Wie bereits erwähnt, ist der Fehler der Kohärenzlänge möglicherweise größer als angegeben.Die Störungsmessung hat gezeigt, wie empfindlich der Sensor auf Schall- bzw. Druckänderungen in der Luft und Erschüterungen ist und wie einfach Störungen erfasst werden können. Entsprechend kann man auch davon ausgehen, dass die Erzeugung der Hologramme sehr empfindlich auf Störungen reagiert hätte. Während der Aufnahme der Hologramme durften wir auch deshalb nicht sprechen (ebenso musste der Raum abgedunkelt werden). Die Hologramme sind, wie beschrieben, im Allgemeinen gut gelungen.
\end{document}