\documentclass[10pt,a4paper]{article}
\usepackage[utf8]{inputenc}
\usepackage[english]{babel}
\usepackage[T1]{fontenc}
\usepackage{amsmath}
\usepackage{amsfonts}
\usepackage{amssymb}
\usepackage{subcaption}
\usepackage{makeidx}
\usepackage{graphicx}
\usepackage{fourier}
\usepackage{listings}
\usepackage{color}
\usepackage{hyperref}
\usepackage[left=2cm,right=2cm,top=2cm,bottom=2cm]{geometry}
\author{Tommy Müller, Marcus Dittrich, Vincent Noculak}
\title{Holografie}


\begin{document}
\section{Fragestellung}

Ziel des Versuchs ist der Aufbau einer Apparatur zur Erzeugung von Weißlichthologrammen. Es soll die Stabilität des Aufbaus überprüft und mehrere Hologramme erstellt werden, sowie die Kohärenzlänge des verwendeten Lasers ermittelt werden.

\section{Durchführung}


\subsection{Aufgabe 4}
Mit der Bragg- Bedingung wir es mit dem gegebenen Zahlenwerten von $\alpha $ = 30° und d = 560nm\\

$n \lambda= 2d \sin(\alpha) $
\\
zu $\lambda$= 560 nm und damit wird das Hologramm gelblich dargestellt.

\end{document}