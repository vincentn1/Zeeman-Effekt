\documentclass[10pt,a4paper]{article}
\usepackage[utf8]{inputenc}
\usepackage[english]{babel}
\usepackage[T1]{fontenc}
\usepackage{amsmath}
\usepackage{amsfonts}
\usepackage{amssymb}
\usepackage{subcaption}
\usepackage{makeidx}
\usepackage{graphicx}
\usepackage{fourier}
\usepackage{listings}
\usepackage{color}
\usepackage{hyperref}
\usepackage[left=2cm,right=2cm,top=2cm,bottom=2cm]{geometry}
\author{Tommy Müller, Marcus Dittrich, Vincent Noculak}
\title{ Mie-Streuung an levitierten Flüssigkeitströpfchen}


\begin{document}

\maketitle
\newpage
\tableofcontents
\newpage

\section{Durchführung}

\subsection{Eichkurve}

Die für den Versuch angezeigte Wechselspannung der Ringelektrode entsprach nicht der tatsächlich angelegten Spannung. Deshalb musste vor der Messung eine Eichkurve gemessen werden. Die gemessene Eichkurve ist in Abbildung \ref{eichkurve1} zu sehen. Es kann ein linearer Zusammenhang zwischen der abgelesenen und tatsächlich angelegten Spannung erkannt werden. Mit der Methode der kleinsten Quadrate wurde eine Ausgleichsgerade durch die Messwerte gelegt. Mit dieser Methode wurde auch der Fehler der Steigung der Gerade abgeschätzt. Die Steigung der Ausgleichsgerade beträgt $0,0824 \pm 0,0002$. Wegen dem linearen Zusammenhang muss beim Ablesen einer Spannung diese nur mit der Steigung der Ausgleichsgerade multipliziert werden, um die angelegte Spannung zu berechnen.

\begin{figure}[h]
	\includegraphics[scale = 0.7]{eichkurve.png}
	\centering
	\caption{Eichkurve der angelegten Wechselspannung an der Ringelektrode; Steigung der Ausgleichsgerade: $0,0824 \pm 0,0002$}
	\label{eichkurve1}
\end{figure}

\section{Versuchsauswertung}

\subsection{Q/m-Werte gefangener Glaskugeln}

Wir haben versucht, geladene Glaskügelchen stabil in der Paul-Falle gefangenzuhalten. Dies ist nur bei zwei Kügelchen für mehr als eine Spannungseinstellung gelungen. Die gemessenen Kugeln können in Tabelle \ref{gefangen} gesehen werden. Zum berechnen des $\frac{Q}{m}$-Verhältnisses der beiden Kugel müssen wir die Spannungen der Deckel- und Bodenelektrode betrachten.

Eine numerische Berechnung des $\vec{E}$-Feldes in der Mitte der Paul-Falle ist aus der Versuchsanleitung vorgegeben als 

\begin{equation}
	E_{Mitte} = \frac{0,798 \cdot (U_{Deckel} - U_{Boden})}{2 z_0}
	\label{emitte1}
\end{equation}

$z_0$ ist der Abstand von der Mitte bis zum Boden/zur Decke der Paul-Falle. Wenn das E-Feld die Gravitationskraft eines Glaskügelchens kompensiert muss gelten

\begin{equation}
	Q E = m g
\end{equation}

Daraus folgt durch Einsetzen von \eqref{emitte1}

\begin{equation}
	\frac{Q}{m} = \frac{2 z_0 g}{0,798 \cdot(U_{Deckel} - U_{Boden})}
	\label{mqformel1}
\end{equation}

 Der Abstand zwischen Boden- und Deckelelektrode ist als $\frac{10}{\sqrt{2}}mm$ gegeben. Somit folgt $z_0 = \frac{5}{\sqrt{2}}mm$.
 
 Der mit \eqref{mqformel1} berechnete Wert von $\frac{Q}{m}$ ist für Kugel Nummer 1 $(4,6 \pm 0,4)*10^{-3} \frac{kg}{C}$. Die Standardabweichung wurde als Fehler genommen. Für Kugel Nummer 2 lässt sich $\frac{Q}{m}$ nicht berechnen, weil Deckel- und Bodenspannung identisch sind. Weil für kleinere Differenzen von $U_{Deckel}$ und $U_{Boden}$ $\frac{Q}{m}$ größer wird, kann man vermuten, dass Kugel Nummer 2 einen großen Wert für $\frac{Q}{m}$ hat. Weil wir nur für eine Kugel den Wert für $\frac{Q}{m}$ bestimmen konnten, ist es uns nicht möglich, Aussagen darüber zu treffen, in welchen Bereichen die Werte für die spezifische Ladung der Glaskügelchen liegen.

\begin{table}[h!]
	\centering
	\begin{tabular}{|l|l|l|l|l|}\hline
		Kugelnummer & $U_{Deckel}$/V ($\pm 1V$)& $U_{Boden}$/V ($\pm 1V$)& $U_{Ring}$/V ($\pm 1V$)& Wechselspannungsfrequenz f/Hz\\\hline
		1 & 158 & 142 & 267 & 101\\
		1 & 161 & 140 & 306 & 101\\
		1 & 160 & 141 & 317 & 101\\
		1 & 161 & 140 & 286 & 101\\
		2 & 150 & 150 & 323 & 101\\
		2 & 150 & 150 & 292 & 101\\\hline
	\end{tabular}
	\caption{Messwerte für in der Paul-Falle gefangene Glaskügelchen}
	\label{gefangen}
\end{table}

\subsection{Stabiler Bereich im a/q-Raum}

Wir betrachten die Werte für a und q, bei denen die Glaskügelchen unserer Messungen stabil waren. Als Stabilitätskriterium der Kügelchen hatten wir festgelegt, dass die Amplitude, in der sie oszillieren, kleiner gleich ihrem zweifachen Durchmesser sein muss. a und q lassen sich mit folgenden Formeln berechnen:

\begin{equation}
	a = -8 \frac{Q}{m} \frac{U}{r_o^2 (2 \pi f)^2}
\end{equation}
\begin{equation}
	q = -4 \frac{Q}{m} \frac{V_0}{r_0^2 (2\pi f)^2}
\end{equation}

Der Durchmesser der Ringelektrode ist mit $10mm$ gegeben. Daraus ergibt sich $r_0 = 5mm$.

Für die zweite Kugel war die spezifische Ladung nicht zu berechnen, weshalb wir nur den stabilen Bereich der ersten Kugel betrachten können. In Tabelle \ref{gefangen2} sind die benötigten gemessenen Werte zur Berechnung von a und q angegeben. Die daraus berechneten Werte sind auch in der Tabelle zu sehen. 

Wir tragen die erhaltenen Werte für a und q in ein Diagramm ein und untersuchen, ob die Werte in dem theoretisch vorhergesagten stabilen Bereich liegen. Die Grenzen des ersten Stabilitätsbereiches des Stabilitätsdiagramms sind durch folgende Kurven gegeben:

\begin{align}
	a(q) = - \frac{1}{2} q^2 + \frac{7}{128} q^4 - \frac{29}{2304} q^6 + \frac{68687}{18874368}\\
	a(q) = 1-q-\frac{1}{8} q^2 + \frac{1}{64} q^3 - \frac{1}{1536} q^4 - \frac{11}{35864} q^5	
\end{align}

Weil die Stabilität nur von Betrag für q anhängt, spiegeln wir unsere q-Werte am Nullpunkt, bevor wir das Diagramm erstellen. 

Das erhaltene Diagramm für unsere Messwerte im Stabilitätsdiagramm kann in Abbildung \ref{stabiler_bereich1} gesehen werden. Alle a/q-Wertepaare liegen innerhalb des theoretisch vorhergesagten stabilen Bereiches. Auch die dreifachen Fehlerintervalle der Messwerte liegen noch im stabilen Bereich. Es fällt auf, dass alle Messwerte sehr zentral und nah aneinander im stabilen Bereich liegen. Experimentell hatten wir beobachtet, dass die Kügelchen sehr schnell labil wurden und aus der Paul-Falle entkamen. Dies ist beim Betrachten der Werte im Diagramm, die zu unserer ersten beobachteten Kugel gehören, widersprüchlich. Laut dem Diagramm hätten wir die angelegte Gleich- oder Wechselspannung noch etwas verändern können, ohne dass die Kugel instabil wird. Das war uns nicht möglich. Folglich erreichten wir mit unseren Messwerten auch nicht die theoretische Grenze des stabilen Bereiches.

Da alle unsere Messwerte im theoretisch vorhergesagten stabilen Bereich liegen, kann angenommen werden, dass dieser im Grunde für unseren Messaufbau gelten muss. Weil es uns nicht möglich war, mit den Messwerten bis zur Grenze des theoretischen stabilen Bereichs zu gelangen ohne, dass die Kügelchen instabil wurden, ist es möglich, dass der tatsächliche stabile Bereich unseres Versuchaufbaus kleiner, als der theoretisch vorhergesagte ist. 

Es muss berücksichtigt werden, dass in die Messdaten nicht zu viel hineininterpretiert werden darf, weil es sich bei ihnen um nur vier Messpunkte handelt. Von vier Messpunkten ausgehend, kann man nur sehr eingeschränkt Aussagen über die Struktur des stabilen Bereiches treffen. 

\begin{table}[h!]
	\centering
	\begin{tabular}{|l|l|l|l|l|l|}\hline
		Kugelnummer & Gleichspannung U/V ($\pm 2V$) & Wechselspannung $V_0$/V ($\pm 1V$)& f/Hz & a & q\\\hline
		1 & 16 & 267 & 101 & $-0,059 \pm 0,009$ & $-0,49 \pm 0,04$\\
		1 & 21 & 306 & 101 & $-0,077 \pm 0,01$ & $-0,56 \pm 0,05$\\
		1 & 19 & 317 & 101 & $-0.069 \pm 0,009$ & $-0.58 \pm 0,05$\\
		1 & 21 & 286 & 101 & $-0,077 \pm 0,01$ & $-0,52 \pm 0,05$\\
		2 & 0 & 323 & 101 &&\\
		2 & 0 & 292 & 101 &&\\\hline
	\end{tabular}
	\caption{Gemessene Spannungen und berechnete Werte für a und q}
	\label{gefangen2}
\end{table}

\begin{figure}[h]
	\includegraphics[scale = 0.7]{stabiler_bereich.png}
	\centering
	\caption{Messwerte für stabile Glaskügelchen im Stabilitätsdiagramm}
	\label{stabiler_bereich1}
\end{figure}

\section{Diskussion}

Es erwies sich als schwierig, geladene Glaskügelchen in der Paul-Falle gefangen zu halten. Nur bei zwei Kügelchen war uns dies über mehrere Messungen gelungen. An diesem Kügelchen bestimmten wir die spezifische Ladungen und, zusammenhängend mit dem Stabilitätsdiagramm der Paul-Falle, die Werte für a und q.
Nur eine der beiden Kugeln erwies sich als geeignet, um $\frac{Q}{m}$, a und q zu untersuchen. Ihr Wert für die spezifische Ladung betrug $(4,6 \pm 0,4)*10^{-3} \frac{kg}{C}$. Dieser Wert lässt sich nicht mit Literaturwerten vergleichen, weil er für jede Kugel zufällig ist. Nur in der Größenordnung der spezifischen Ladung könnten die Kugeln übereinstimmen.

Wir untersuchten stabile Bereiche für die Glaskügelchen in der Paul-Falle und fanden mithilfe des Diagramms in Abbildung \ref{stabiler_bereich1} heraus, dass unsere gemessenen Werte für stabile Glaskügelchen dem theoretischen Stabilitätsdiagramm nicht widersprechen. Wegen der geringen Anzahl an Messpunkten konnten wir die Form des Stabilitätsdiagramms experimentell nicht überprüfen. Weil keine Messpunkte am Rand des theoretischen stabilen Bereichs gemessen werden konnten, ist es möglich, dass der tatsächliche stabile Bereich im Experiment kleiner, als in der Theorie ist. Dies könnte auf Fehlerquellen im Experiment zurückzuführen sein. Mögliche Ursachen wären ein nicht stabiler Messaufbau oder Elektroden, dessen Form von der theoretischen Paul-Falle abweichen. Auch muss berücksichtigt werden, dass die Glaskügelchen nicht punktförmig sind. Dadurch könnte das Stabilitätsdiagramm anders aussehen.

\end{document}