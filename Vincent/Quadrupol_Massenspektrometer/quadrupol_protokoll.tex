\documentclass[10pt,a4paper]{article}
\usepackage[utf8]{inputenc}
\usepackage[english]{babel}
\usepackage[T1]{fontenc}
\usepackage{amsmath}
\usepackage{amsfonts}
\usepackage{amssymb}
\usepackage{subcaption}
\usepackage{makeidx}
\usepackage{graphicx}
\usepackage{fourier}
\usepackage{listings}
\usepackage{color}
\usepackage{hyperref}
\usepackage[left=2cm,right=2cm,top=2cm,bottom=2cm]{geometry}
\author{Tommy Müller, Marcus Dittrich, Vincent Noculak}
\title{Rastertunnelmikroskopie}

\lstset{language=C++,
	keywordstyle=\bfseries\color{blue},
	commentstyle=\itshape\color{red},
	stringstyle=\color{green},
	identifierstyle=\bfseries,
	frame=single}
\begin{document}

\maketitle
\newpage
\tableofcontents
\newpage

\section{Theoretische Vorbereitung}

\subsection{Quadrupolfeld}

Das elektrische Potential einer Ladungsverteilung aus Punktladungen ist

\begin{equation}
	\phi(\textbf{r}) = \frac{1}{4 \pi \epsilon_0 \sum_{j} \frac{q_j}{|\textbf{r}-\textbf{r_j}|}}
\end{equation}

\end{document}